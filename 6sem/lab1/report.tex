\documentclass[11pt, oneside, final]{article} 
\usepackage[utf8]{inputenc} 
\usepackage{a4wide} 
\usepackage[russian]{babel} 
\usepackage{graphicx} 
\usepackage{epstopdf} 
\usepackage{amsmath} 
\usepackage{amsfonts} 
\usepackage{amssymb} 
\usepackage{amsthm}
\usepackage{epstopdf}
\usepackage[perpage]{footmisc}
%\newtheoremstyle{mytheorem}% name of the style to be used
%  {}% measure of space to leave above the theorem. E.g.: 3pt
%  {}% measure of space to leave below the theorem. E.g.: 3pt
%  {}% name of font to use in the body of the theorem
%  {}% measure of space to indent
%  {}% name of head font
%  {:}% punctuation between head and body
%  { }% space after theorem head; " " = normal interword spacewww
%  {}% Manually specify head
%\theoremstyle{mytheorem}
\newtheoremstyle{break}%
  {}{}%
  {\itshape}{}%
  {\bfseries}{}%  % Note that final punctuation is omitted.
  {\newline}{}
\theoremstyle{break}
\newtheorem*{PMP}{Принцип Максимума Понтрягина} 
\numberwithin{equation}{section} 
\theoremstyle{plain}
\newtheorem{theorem}{Теорема}[section] 
\newtheorem{property}{Свойство}[section] 
\newtheorem{corollary}{Следствие}[theorem] 
\newtheorem{lemma}[theorem]{Лемма} 
\newtheorem*{statement}{Утверждение} 
\theoremstyle{definition}
\newtheorem{definition}{Определение}[section] 
\renewenvironment{proof}{
\noindent\textit{Доказательство: }} {\qed}
\newcounter{icount}
\graphicspath{{figures/f1/}{figures/f2/}{figures/f3/}{figures/f4/}}
%commands
\newcommand \bitem[1][]{
\item \textbf{#1}} 
\newcommand \four[1][\lambda]{\mathfrak{F}(#1)} 
\newcommand \fft[1][\lambda]{F(#1)} 
\newcommand \rarrow{\rightarrow} 
\newcommand \real{\mathbb{R}}
\newcommand \intinf[1][{\,dt}]{ \int\limits_{-\infty}^{+\infty}{{#1}}} 
\renewcommand \qed{$\blacksquare$}
\newcommand{\scalar}[2]{\left< #1, #2\right>}

\DeclareMathOperator{\sgn}{sgn}

\begin{document}

%Title
    \thispagestyle{empty}
    \begin{center}
        \ \vspace{-3cm}
    
        \includegraphics[width=0.5
        \textwidth]{msu}\\
        {\scshape Московский государственный университет имени М.~В.~Ломоносова}\\
        Факультет вычислительной математики и кибернетики\\
        Кафедра системного анализа
    
        \vfill
    
        {\LARGE Отчёт по практикуму}
    
        \vspace{1cm}
    
        {\Huge\bfseries "<Линейная Задача Быстродействия">} 
    \end{center}

    \vspace{1cm}
    \begin{flushright}
        \large \textit{Студент 315 группы}\\
        В.\,А.~Сливинский
    
        \vspace{5mm}
    
        \textit{Руководитель практикума}\\
        к.ф.-м.н., доцент П.\,А.~Точилин 
    \end{flushright}

    \vfill
    \begin{center}
        Москва, 2018 
    \end{center}
    \pagebreak

    %Contents
    \tableofcontents

    \pagebreak


    %Task

    \section{Постановка~задачи}
    \label{sec:task}
    \subsection{Общая~формулировка~задачи} 
    \label{sub:general}
    Задана линейная система ОДУ: 
    \begin{equation} 
        \label{task:system} 
        \dot x = Ax + u + f,\:t \in [t_0, +\infty)
    \end{equation}
    Здесь, \(x, f \in \real^2, \, A \in \real^{2\times2}, \, u \in \real^2 \).
     Кроме того, на управление \( u \) наложено дополнительное ограничение \(u \in \mathcal{P} \). Пусть \( \mathcal{X}_0 \)~---~начальное множество значений фазового вектора, \( \mathcal{X}_1 \)~---~целевое множество значений фазового вектора. Для заданных множеств \( \mathcal{X}_0 ,\, \mathcal{X}_1, \, \mathcal{P} \) необходимо решить задачу быстродействия, т.е. найти минимальное время \(T > 0 \), за которое траектория системы, выпущенная в момент времени \( t_0 \) из некоторой точки множества \( \mathcal{X}_0 \), может попасть в некоторую точку множества \( \mathcal{X}_1 \). 
    \begin{gather}
        \label{task:P} 
        \mathcal{P} = p + \left\{ (x_1, x_2) \in \real^2 \, \colon 9x_1^2 + 4x_2^2 \leqslant r \right\}, \; p \in \real ^2; \\
        \label{task:X0}
        \mathcal{X}_0 = \{x_0\}; \\
        \label{task:X1}
        \mathcal{X}_1 = \left\{ x = (x_1, x_2) \in \real^2 \, \colon a(x_1 - x_{11})^2 + b |x_2 - x_{12}| \leqslant c \right\}, \; a, b, c > 0.
    \end{gather} 
    Требуется: 
    \begin{enumerate} 
        \item Написать в среде Matlab программу с пользовательским интерфейсом, которая по заданным значениям параметров \(A, f, t_0, r, p, x_0, a, b, c, x_{11}, x_{12}\) определяет, разрешима ли задача \eqref{task:system}. Если задача разрешима, программа должна (приближённо) найти значение T и построить графики компонент оптимального управления, оптимальной траектории, сопряжённых переменных. Кроме того, программа должна допускать возможность улучшения решения, как локальным, так и глобальным методами.
        \item Для различных значений параметров (в том числе, для различных собственных значений матрицы A) провести анализ системы \eqref{task:system}, численно решить задачу и построить соответствующие графики.
    \end{enumerate}
    \pagebreak
    %Formal task
    \subsection{Формальная~постановка~задачи} 
    \label{sub:formal}
    \begin{enumerate}
        \item Провести необходимые исследования системы \eqref{task:system} и привести сопутствующие теоретические выкладки;
        \item \label{enum:method} Разработать и описать численный метод решения задачи и возникающих подзадач;
        \item Реализовать на языке MATLAB программу, удовлетворяющую условиям из \ref{sub:general} и реализующую численный метод из пункта \ref{enum:method}.
        Для этого, реализовать:
        \begin{itemize}
            \item  Пользовательский интерфейс ввода исходных данных;
            \item  Алгоритм поиска управлений и траекторий, подозрительных на оптимальные, а также алгоритм отбора из них оптимальных (при наличии таковых);
            \item  Алгоритм и интерфейс построения требуемых графиков;
            \item  Алгоритм локального и глобального улучшения решения; 
            \item  Алгоритм сохранения и загрузки промежуточных данных, значений параметров и полученных ответов.
        \end{itemize}
        \item Построить, используя написанную программу, графики для различных значений параметров и проанализировать полученные решения.
    \end{enumerate}
    \pagebreak
    \section{Некоторые~необходимые~теоретические~выкладки}
    \label{sec:theory}
    Прежде всего, установим принцип максимума Понтрягина в следующей формулировке:\footnote{Доказательство приведено, например, в \cite{Pontr'yaginEtAl:maximum}}
    \begin{PMP}[В формулировке из \cite{RoublevTochilin:matlab}]
        \label{th:max}
        Пусть \((u^{*}(\cdot), x^{*}(\cdot))\)~--- оптимальная пара. Тогда существует \(\psi(t) \in AC[t_0, t_1],\: t \in [t_0, t_1]\!: \)
        \begin{align}
            \dot \psi =& -\!A^{T}\psi \label{eq:conj} \\
            \scalar{Bu^{*}(t)}{\psi(t)} =& \, \rho(\psi(t)|B\mathcal{P}) \label{eq:optimum} \\ 
            \scalar{\psi(t_0)}{x^{*}(t_0)} =& \, \rho(\psi(t_0)|\mathcal{X}_0) \label{eq:trans:1} \\
            \scalar{-\psi(t_1)}{x^{*}(t_1)} =& \, \rho(-\psi(t_1)|\mathcal{X}_1) \label{eq:trans:2}            
        \end{align}
    \end{PMP}
    \noindent Систему \eqref{eq:conj} называют \emph{сопряжённой системой}, её решение \(\psi = \psi(t)\)~--- \emph{сопряжёнными переменными}, а условия \eqref{eq:trans:1} и \eqref{eq:trans:2}~--- \emph{условиями трансверсальности}. Условие \eqref{eq:optimum} позволяет выделить из всех возможных управлений семейство "<подозрительных">  на оптимальные. \\
    Для того, чтобы однозначно определить решение системы \eqref{eq:conj}, нам необходимо присовокупить к ней некоторые начальные условия. В результате получим задачу Коши
    для сопряжённой системы:
    \begin{equation}
        \left\{
        \label{eq:cauchy}
        \begin{aligned}
            & \dot \psi(t) = -\!A^{T}\psi(t), \ t \in [t_0, t_1] \\
            &\left[
            \begin{aligned}
                & \psi(t_0) = \psi_0 \\
                & \psi(t_1) = \psi_1
            \end{aligned}
            \right.
        \end{aligned}
        \right.
    \end{equation} 
    Условие \eqref{eq:optimum}, в силу свойств скалярного произведения, можно переписать в следующем виде:
    \begin{equation*}
        \scalar{B^{T}\psi(t)}{u^{*}(t)} = \, \rho(B^{T}\psi(t)|\mathcal{P}) \label{eq:optimum2}  
    \end{equation*}
    В свою очередь, раскрыв определение опорной функции множества \(\mathcal{P}\) в направлении \(B^{T}\psi(t)\), окончательно получим:
    \begin{equation}
        \label{eq:optimum:final}
        \scalar{B^{T}\psi(t)}{u^{*}(t)} = \, \sup_{u(t)\in\mathcal{P}}\scalar{B^{T}\psi(t)}{u(t)}
    \end{equation}
    Заметим, что множество \(\mathcal{P}\) (см. \eqref{task:P}) есть эллипсоид \( \mathcal{E}(p, P)\), где \(P = 
    \left(\begin{smallmatrix} \frac{1}{3} & 0 \\ 0 & \frac{1}{2} \end{smallmatrix}\right)\)~--- матрица конфигурации.
    Из \cite{Roublev:optimal:linear} известно, что решение уравнения \eqref{eq:optimum:final} \(u^{*}(t)\) представимо в виде:
    \begin{equation}
    \label{eq:control}
    u^{*}(t) = p + \dfrac{PB^{T}\psi(t)}{\sqrt{\scalar{B^{T}\psi(t)}{PB^{T}\psi(t)}}}
    \end{equation}
    Данное выражение корректно при \(B \neq 0\), так как из \cite{Roublev:optimal:nonlinear} и \cite{Pontr'yaginEtAl:maximum} известно, что \(\psi(t) \neq 0\) для любого допустимого \(t\), а \(P \neq 0\).
    \section{Написание функции \texttt{plotFT} } % (fold)
    \label{sec:programm}
    \subsection{Разбиение на подзадачи} % (fold)
    \label{sub:tasks}
    Написание функции \texttt{plotFT} удобно делать по частям, разбив поставленнию задачу на следующие подзадачи
    \begin{enumerate}
        \item Вычисление аппроксимации преобразования Фурье
        \item Подготовка фигуры к выводу графиков
        \item Вывод графиков быстрого и, при необходимости, аналитического преобразований Фурье 
    \end{enumerate}
    Соответственно, будем решать подзадачи в приведённом порядке, приводя необходимые выкладки и теоретические обоснования\footnote{Полный код функции \texttt{plotFT} приведён в приложении~\ref{lst:code} (стр.~\pageref{lst:code})}.
    % subsection tasks (end)
    \subsection{Вычисление аппроксимации преобразования Фурье} % (fold)
    \label{sub:fft}
    \begin{enumerate}
        \item 
        Найдем число вычисляемых узлов сеточной функции \texttt{nPoints}, хранимое в переменной \texttt{n} по формуле~\eqref{eq:npoints}:
        \begin{verbatim}
            a = inpLimVec(1);
            b = inpLimVec(2);
            n = floor((b - a) ./ step) + 1;
        \end{verbatim}
        \item 
        Откорректируем значение шага \texttt{step} в соответствии с числом точек (формула~\eqref{eq:step}):
            \begin{verbatim}
                step = (b - a) ./ (n - 1);
            \end{verbatim}
        \item 
        Вычислим на сетке \([a, b]\), состоящей из \texttt{n} точек значения самой функции \(f(t)\), тем самым получим дискретизацию \(f_\text{дискр}(t)\),
        затем воспользуемся функциями MATLAB \texttt{fft()} и \texttt{fftshift()}, первая из которых вычисляет дискретное преобразование Фурье (\textbf{ДПФ}) 
        функции \(f_\text{дискр}(t)\), однако возвращает вектор значений в зеркальном виде, а вторая ~--- "<отзеркаливает"> этот вектор, приводя его к нормальному виду\footnote{Здесь и далее все использованные средства языка \texttt{MATLAB} и спецификации взяты из \cite{RoublevTochilin:matlab} и \cite{Matlab:help}}. 
        Искомая аппроксимация преобразования Фурье \(F(\lambda)\) вычисляется по следующей формуле (доказательство её справедливости приведено в \cite{Roublev:fourier}):
        \begin{equation} \label{eq:from_discr}
            F(\lambda) = \mathtt{step} \cdot F_\text{дискр}(\lambda)
        \end{equation}
        Здесь \( F_\text{дискр}(\lambda) \) ~--- вектор значений \textbf{ДПФ} функции \(f_\text{дискр}(t)\), полученный путем применения 
        \texttt{fftshift(fft(\dots))} к вектору значений \(f_\text{дискр}(t)\) на заданной сетке.\\
        Приведём, в заключение, общую схему работы данного этапа:
        \[
            f(t) \xrightarrow[\text{на сетке}]{\text{дискретизация}} f_\text{дискр}(t)
            \xrightarrow{\mathtt{fftshift(fft())}} F_\text{дискр}(\lambda) 
            \xrightarrow{\eqref{eq:from_discr}} F(\lambda)
        \]
        \clearpage
        \item
        Преобразование Фурье рассматривается на отрезке \small{\(\left[\frac{-\pi}{\Delta t}, \frac{\pi}{\Delta t}\right]\)}, длины \(2\pi/\Delta t\), 
        разбитом на \texttt{nSteps} точек. Обратившись к \cite{Roublev:fourier}, установим следующее свойство преобразования Фурье: 
        \[
            \boxed{f(t - t_0) \rarrow e^{-i\lambda t_0}F(\lambda)}
        \]
        Соответственно, для получения желаемого результата, полученный вектор значений \textbf{ДПФ} следует домножить на соответствующие значения экспоненты.
    \end{enumerate}
    \subsection{Подготовка фигуры к выводу графиков} % (fold)
    \label{sub:prepare_graphs}
    В поле \texttt{UserData} фигуры \texttt{fHandle} будем хранить \texttt{handle} двух соответствующих осей (\texttt{axes}),
    а также окно вывода по оси абсцисс \(\lambda\). В случае, если \texttt{UserData} у поданной фигуры пуст, сформируем его, построив две оси для вещественной и мнимой частей
    преобразования Фурье, соответствующим образом выбирая окно вывода: для этого просматриваем вектор значений \textbf{ДПФ} и находим левую и правую границы, на которых значение
    превышает некоторое \(\varepsilon\), обозначенное в программе как \texttt{moe} (англ. \emph{margin of error}). 
    При наличии у фигуры поля \texttt{UserData}, но отсутствии в нём осей и/или пределов, дополним недостающие поля аналогичным образом.
    Наконец, сформированную структуру запишем в поле \texttt{UserData} фигуры \texttt{fHandle}.
    \subsection{Вывод графиков} % (fold)
    \label{sub:plotting}
    Вывод графиков осуществляется стандартными средствами языка MATLAB, при этом, если поле \texttt{fFTHandle} не пусто, то выводится и график функции, на которую
    указывает \texttt{fFTHandle}.
    % subsection plotting (end)
    % subsection \xD0\xBF\xD0\xBE\xD0\xB4\xD0\xB3\xD0\xBE\xD1\x82\xD0\xBE\xD0\xB2\xD0\xBA\xD0\xB0 (end)
    % section programm (end)
    %fourier
    \clearpage
       \section{Вычисление~аналитических~преобразований~Фурье}

       \subsection{Некоторые~необходимые~обозначения~и~соотношения}
       Напомним, что преобразование Фурье \( \four \) функции \(f(t)\) задаётся формулой \eqref{fourier_transform}: 
       \begin{align*}
           \four = \intinf[ {f(t) e^{-i\lambda t}\, dt}] 
       \end{align*}
       Впредь, будем для краткости писать:
       \[ \boxed{ f(t)\rarrow\four} \]
       Напомним также следующие свойства преобразования Фурье:
       \begin{property}\label{property:linear} 
           \mdseriesПусть 
           \begin{gather*}
               f(t) = \alpha \cdot f_1(t) + \beta \cdot f_2(t) \text{, и } \left\{ 
               \begin{aligned}
                   f_1(t) &\rarrow \mathfrak{F_1}(\lambda) \\
                   f_2(t) &\rarrow \mathfrak{F_2}(\lambda) 
               \end{aligned}
               \right. 
           \end{gather*}
           \\
           Тогда:
           \[ f(t) \rarrow \alpha \cdot \mathfrak{F_1}(\lambda) + \beta \cdot \mathfrak{F_2}(\lambda) \]
       \end{property}
       \begin{property}\label{property:product} 
           \mdseriesПусть 
           \begin{gather*}
               f(t) = f_1(t) \cdot f_2(t) \text{ , и } \left\{ 
               \begin{aligned}
                   f_1(t) &\rarrow \mathfrak{F_1}(\lambda) \\
                   f_2(t) &\rarrow \mathfrak{F_2}(\lambda) 
               \end{aligned}
               \right. 
           \end{gather*}
           \\
           Тогда:
           \[ 2\pi f_1(t) \cdot f_2(t) \rarrow (\mathfrak{F_1 * F_2)}(\lambda) \text{ , где }(\mathfrak{F_1 * F_2})(\lambda) = \intinf[{\bigl[\mathfrak{F_1}(\lambda - s) \cdot \mathfrak{F_2}(s)\bigr] \, ds}] \]
       \end{property}
       \noindentОтметим некоторые тривиальные\footnote{Вывод этих преобразований, а также доказательства свойств~\eqref{property:linear} и~\eqref{property:product} можно найти в \cite{Roublev:fourier}} преобразования Фурье: 
       \begin{flalign}\label{fourier:delta}
           \delta(\lambda) &\rarrow 1 \\
           \label{fourier:1} 1 &\rarrow 2\pi\delta(\lambda) \\
           \label{fourier:exp} e^{iat} &\rarrow 2\pi\delta(\lambda - a) \\
           \label{fourier:cos} \cos(t) = \frac{e^{it} + e^{-it}}{2} & \rarrow \pi(\delta(\lambda - 1) + \delta(\lambda + 1)) \\
           \label{fourier:1/t} \dfrac{1}{t} &\rarrow -i \pi \sgn(t) 
       \end{flalign}
       Где \( \delta(t) = 
       \begin{cases}
           +\infty,& t = 0 \\
           0,& t \not= 0 
       \end{cases}
       \)~--- дельта-функция Дирака, а соотношение \eqref{fourier:cos} вытекает из свойства~\ref{property:linear}, с учётом \eqref{fourier:exp}. \\
       %delta
       Установим также важное отношения для свёртки дельта-функции с произвольной функцией \(\varphi(t)\): 
       \begin{align}\label{delta:conv}
           \boxed{\left(\delta * \varphi \right) (s) = \intinf[{\delta(s - T) \cdot \varphi(T)\, dT} = \varphi(s)]} 
       \end{align}

       \noindent Докажем следующее соотношение: 
       \begin{lemma}
           \begin{equation}\label{fourier:exp_abs} 
               e^{-A|t|} \rarrow \dfrac{2A}{A^2 + \lambda^2} 
           \end{equation}
       \end{lemma}
       \begin{proof}
           \[ 
           \begin{split}
               \intinf[{e^{-A|t|} \cdot e^{-i\lambda t} \, dt}] &= \int\limits_{-\infty}^0{e^{(A - i\lambda) t} \, dt} + \int\limits_0^{+\infty}{e^{-(A + i\lambda) t} \, dt} = \\
               &= \left[ e^{(A - i\lambda)t} \cdot \frac{1}{A - i\lambda} \right]_{t = -\infty}^{0} - \left[ e^{-(A + i\lambda)t} \cdot \frac{1}{A + i\lambda} \right]_{t = 0}^{+\infty} = \\
               &=\frac{1}{A - i\lambda} + \frac{1}{A + i\lambda} = \frac{2A}{A^2 + \lambda^2} 
           \end{split}
           \]
       \end{proof}

       %Fourier 1
   \clearpage
       \subsection{Вычисление~аналитического~преобразования~Фурье\\функции~\(f_1(t) = e^{-2|t|} \cos(t)\)}

       Преобразование Фурье \( \mathfrak{F_1} (\lambda)\) функции \(f_1(t) = e^{-2|t|} \cos(t) \) задаётся формулой:
       \[ \mathfrak{F_1} (\lambda) = \intinf[{e^{-2|t|} \cos(t) e^{-i\lambda t}\, dt}] \]
       \begin{statement}
           \begin{equation}\label{eq:fourier:f1} 
               \boxed{ \mathfrak{F_1}(\lambda) = \dfrac{4(\lambda^2 + 5)}{\lambda^4 + 6\lambda^2 + 25} } 
           \end{equation}
       \end{statement}
       \begin{proof}
           Заметим, что \(f_1(t) \) представима в виде: 
           \begin{equation}\label{factor:f1} 
               f_1(t) = g_1(t) \cdot g_2(t) \text{, где }g_1(t) = e^{-2|t|},\,g_2(t) = \cos(t) 
           \end{equation}
           Пользуясь этим соотношением, выражениями для преобразований Фурье \(g_1(t)\) \eqref{fourier:exp_abs} и \(g_2(t)\) \eqref{fourier:cos}, установленным свойством \ref{property:product} и соотношением \eqref{delta:conv} для свёртки с дельта-функцией, получим:
           \[ 
           \begin{split}
               \mathfrak{F_1} (\lambda) &= \dfrac{1}{2\pi} \intinf[{\dfrac{4}{4 + T^2} \cdot \pi(\delta(\lambda - T- 1) + \delta(\lambda + 1 - T))\,dT}] = \\
               &=\dfrac{2}{4 + (\lambda - 1)^2} + \dfrac{2}{4 + (\lambda + 1)^2} = \dfrac{4(\lambda^2 + 5)}{\lambda^4 + 6\lambda^2 + 25} 
           \end{split}
           \]
       \end{proof}
       \clearpage
       %Fourier 2

       \subsection{Вычисление~аналитического~преобразования~Фурье\\функции~\(f_2(t) = \frac{e^{-|t|} - 1}{t} \)}
       Преобразование Фурье \( \mathfrak{F_2} (\lambda)\) функции \(f_2(t) = \frac{e^{-|t|} - 1}{t} \) задаётся формулой:
       \[ \mathfrak{F_2} (\lambda) = \intinf[{\dfrac{e^{-|t|} - 1}{t} e^{-i\lambda t}\, dt}] \]
       \begin{statement}
           \begin{equation}\label{eq:fourier:f2} 
               \boxed{ \mathfrak{F_2}(\lambda) = i\left(\pi\sgn(\lambda) - 2\arctg(\lambda)\right) } 
           \end{equation}
       \end{statement}
       \begin{proof}
           Аналогично \eqref{factor:f1} представим \(f_2(t) \) в виде: 
           \begin{equation}\label{factor:f2} 
               f_2(t) = g_1(t) \cdot g_2(t) \text{ где } g_1(t) = \left(e^{-|t|} - 1\right)\text{, } g_2(t) = \dfrac{1}{t} 
           \end{equation}
           Пользуясь установленными свойствами~\ref{property:linear},~\ref{property:product}, выражениями для преобразований Фурье~\(g_1(t)\) \eqref{fourier:exp_abs},~\eqref{fourier:1} и~\(g_2(t)\)~\eqref{fourier:1/t} и соотношением~\eqref{delta:conv} для свёртки с дельта-функцией, получим:
           \[ 
           \begin{split} 
               f_2(t) \rarrow \mathfrak{F_2} (\lambda) &= \intinf[{ \dfrac{e^{-|t|} - 1}{t} e^{-i\lambda t}\, dt}] = 
               \dfrac{1}{\pi}\left[\left({\dfrac{1}{1 + (\cdot)^2} - \pi\delta(\cdot)}\right) * \left({-i \pi \sgn(\cdot)}\right)\right] \negmedspace({\lambda}) = \\
               &= -i\int\limits_{\lambda}^{+\infty}{\dfrac{1}{1 + T^2}\, dT} + i\int\limits_{-\infty}^{\lambda}{\dfrac{1}{1 + T^2}\, dT} + \pi i \sgn(\lambda) = \\
               &= i\left(\pi \sgn(\lambda) - 2\arctg(\lambda)\right) 
           \end{split}
           \]
       \end{proof}
       \clearpage
    % section code (end)
    \begin{thebibliography}{0}
        \addcontentsline{toc}{section}{Список литературы}
        \bibitem{Roublev:optimal:linear} И. В.~Рублёв. \emph{Лекционный курс Оптимальное Управление (Линейные Системы)},
        кафедра Системного~Анализа, Факультет Вычислительной Математики и Кибернетики, МГУ~им.~М.~В.~Ломоносова, 
        2017
        \bibitem{Roublev:optimal:nonlinear} И. В.~Рублёв. \emph{Лекционный курс Оптимальное Управление (Нелинейные Системы)},
        кафедра Системного~Анализа, Факультет Вычислительной Математики и Кибернетики, МГУ~им.~М.~В.~Ломоносова, 
        2018
        \bibitem{RoublevTochilin:matlab} Точилин~П.~А. \emph{Лекционный курс Программирование на языке \texttt{MATLAB}},
        кафедра Системного~Анализа, Факультет Вычислительной Математики и Кибернетики, МГУ~им.~М.~В.~Ломоносова, 
        2017~-- 2018
        \bibitem{Pontr'yaginEtAl:maximum} Понтрягин Л. С., Болтянский В. Г., Гамкрелидзе Р. В., Мищенко Е. Ф.~\emph{Математическая~теория~оптимальных~процессов}, — М.: Наука, 1976.
        \bibitem{Matlab:help} Справочные средства языка \texttt{MATLAB}
    \end{thebibliography}
\end{document}
