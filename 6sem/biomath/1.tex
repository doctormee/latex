\section{Динамические системы с дискретным временем}
\subsection{Постановка задачи}
Дана следующая многомерная динамическая система с дискретным временем:
\begin{equation}
    \label{sys1}
    \left\{
    \begin{aligned}
        &u_{t+1} = au_t(1 - u_t) - u_tv_t&\\
        &v_{t+1} = \dfrac{1}{b}u_tv_t&
    \end{aligned}
    \right.u, v, a, b > 0
\end{equation}
Для системы~\eqref{sys1} требуется:
\begin{enumerate}
    \item Найти неподвижные точки 
    \item Исследовать неподвижные точки на устойчивость
    \item Построить бифуркационную диаграмму
    \item Проверить существование циклов длины 2 и 3
    \item Проверить существование бифуркации Неймарка--Сакера и в случае её обнаружение построить инвариантную кривую
\end{enumerate}
\subsection{Биологическая интерпретация задачи}
Система~\eqref{sys1} представляет собой модель "<Хищник--Жертва">; в ней \(u_t\) это относительная численность жертв в момент времени \(t\) (отношение числа жертв к максимально возможной, определяемой потенциальной ёмкостью экосистемы), \(v_t\) --- относительная численность хищников, параметр \(a\) определяет скорость роста популяции жертв в отсутствии хищника (рождаемость или условный "<естественный прирост">), а параметр \(b\) обратно пропорционален выгоде хищников. Численность жертв в отсутствии хищников описывается дискретным логистическим уравнением \(u_{t+1} = au_t(1 - u_t)\), влияние хищников описывается билинейной функцией \(u_tv_t\), а в отсутствии пищи хищники вымирают за одно поколение (одну единицу времени).
\subsection{Некоторые разумные ограничения на параметры системы}
В главе 3 \cite{Bratus} при исследовании дискретного логистического уравнения была получена следующая оценка для параметра \(a\):
\begin{equation}
    \label{a_bound}
    0 < a < 4
\end{equation}
Помимо этого, исходя из условий неотрицательности траекторий, из первого уравнения системы~\eqref{sys1} выводятся следующие ограничения:
\begin{gather}
    \label{u_bound}
    0 < u_t < 1\\
    \label{v_bound}
    0 < v_t < \dfrac{au_t(1 - u_t)}{u_t}
\end{gather}
\subsection{Поиск неподвижных точек}
Для начала, введём понятие неподвижной точки для многомерной дискретной динамической системы:
\begin{definition}\label{def:stationary} Пусть дана дискретная динамическая система, определяемая отображением \(f\):
    \begin{equation}
        \label{eq:basic_sys}
        u \mapsto f(u) = f(u, r) \quad u \in \mathbb{R}^n,\quad f \in \mathbb{R}^n: \mathbb{R}^n \rightarrow \mathbb{R}^n,\quad r \in \mathbb{R}^m, \quad n,m \in \mathbb{N}
    \end{equation}
    Тогда, точка \(u^*\) называется \emph{неподвижной точкой} системы~\eqref{eq:basic_sys}, если:
    \begin{equation*}
        u^* = f(u^*)
    \end{equation*}
\end{definition}
\begin{note} В системе~\eqref{eq:basic_sys} вектор \(r\)~--- вектор параметров системы
\end{note}
\noindent
Для системы~\eqref{sys1} \(n = m = 2, u = (u, v), f = (f_1, f_2)\), где 
\begin{align*}
    &f_1(u) = au_t(1 - u_t) - u_tv_t \\
    &f_2(u) = \dfrac{1}{b}u_tv_t
\end{align*}
Для нахождения неподвижных точек системы~\eqref{eq:basic_sys} достаточно разрешить по (\(u_t, v_t\)) следующую систему:
\begin{equation}
   \label{sys1:stationary}
   \left\{
   \begin{aligned}
       &u_t = au_t(1 - u_t) - u_tv_t&\\
       &v_t = \dfrac{1}{b}u_tv_t&
   \end{aligned}
   \right.
\end{equation}
Приступим к решению системы~\eqref{sys1:stationary}:
\begin{equation*}
    \left\{
    \begin{aligned}
        &u_t = au_t(1 - u_t) - u_tv_t&\\
        &v_t = \dfrac{1}{b}u_tv_t&
    \end{aligned}
    \right.
    \Longleftrightarrow 
    \left\{
    \begin{aligned}
        &u_t(-au_t + (a - v_t - 1)) = 0\\
        &v_t(\dfrac{u_t}{b} - 1) = 0
    \end{aligned}
    \right.
\end{equation*}
Второе уравнение системы обращается в ноль при \(v_t = 0\) и \(u_t = b\). Исследуем первое уравнение системы для этих значений:
\begin{equation*}
    
\end{equation*}