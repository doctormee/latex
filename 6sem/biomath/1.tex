\section{Динамические системы с дискретным временем}
\subsection{Постановка задачи}
Дана следующая многомерная динамическая система с дискретным временем:
\begin{equation}
    \label{sys}
    \left\{
    \begin{aligned}
        &u_{t+1} = au_t(1 - u_t) - u_tv_t&\\
        &v_{t+1} = \dfrac{1}{b}u_tv_t&
    \end{aligned}
    \right.u, v, a, b > 0
\end{equation}
Для системы~\eqref{sys} требуется:
\begin{enumerate}
    \item Найти неподвижные точки 
    \item Исследовать неподвижные точки на устойчивость
    \item Построить бифуркационную диаграмму
    \item Проверить существование циклов длины 2 и 3
    \item Проверить существование бифуркации Неймарка--Сакера и в случае её обнаружение построить инвариантную кривую
\end{enumerate}
\subsection{Биологическая интерпретация задачи}
Система~\eqref{sys} представляет собой модель "<Хищник--Жертва">; в ней \(u_t\) это относительная численность жертв в момент времени \(t\) (отношение числа жертв к максимально возможной, определяемой потенциальной ёмкостью экосистемы), \(v_t\) --- относительная численность хищников, параметр \(a\) определяет скорость роста популяции жертв в отсутствии хищника (рождаемость или условный "<естественный прирост">), а параметр \(b\) обратно пропорционален выгоде хищников. Численность жертв в отсутствии хищников описывается дискретным логистическим уравнением \(u_{t+1} = au_t(1 - u_t)\), влияние хищников описывается билинейной функцией \(u_tv_t\), а в отсутствии пищи хищники вымирают за одно поколение (одну единицу времени).
\subsection{Некоторые разумные ограничения на параметры системы}
В главе 3 \cite{Bratus} при исследовании дискретного логистического уравнения была получена следующая оценка для параметра \(a\):
\begin{equation}
    \label{a_bound}
    0 < a < 4
\end{equation}