\documentclass[11pt, oneside, final]{article} 
\usepackage[utf8]{inputenc} 
\usepackage{a4wide} 
\usepackage[russian]{babel} 

\begin{document}
    \begin{verbatim}
function [ res ] = solveDirichlet( fHandle, xiHandle, etaHandle, mju, M, N )
    x = linspace(0, 1, M);
    xStep = 1 ./ M;
    y = linspace(0, 1, N);
    yStep = 1 ./ N;
    n = 1:N;
    m = 1:M;
    theta = @(k, l) 1 ./ ((-4 ./ xStep.^2) .* sin(pi .* (k - 1) ./ M) .^2 + ...
        (-4 ./ yStep.^2) .* sin(pi .* (l - 1) ./ N) .^2 - mju);
    phi = zeros(M, N);
    [yGrid, xGrid] = meshgrid(y(2:end), x(2:end));
    [nGrid, mGrid] = meshgrid(n, m);
    Theta = theta(mGrid, nGrid);
    phi(2:end, 2:end) = fHandle(xGrid, yGrid); %now we know all phies, except first ones
    Phi0 = fft2(phi);
    %here we try to find phi(0, i), phi (i, 0) and phi(0, 0) via solving a
    %SLAE:
    %this would be useful later on:
    indicesM = repmat([1:M], M, 1);
    indicesM = abs(indicesM - indicesM.') + 1;

    indicesN = repmat([1:N-1], N - 1, 1);
    indicesN = abs(indicesN - indicesN.') + 1;

    thetaIfft2 = ifft2(Theta);
    thetaFft2 = fft2(Theta);

    firstColI = thetaIfft2(:, 1);
    firstCol = (1./ (M * N)) .* thetaFft2(:, 1);
    firstRowI = thetaIfft2(1, :).';
    firstRow = (1./ (M * N)) .* thetaFft2(1, :).';

    A12 = ifft(fft(Theta, [], 2), [], 1);
    A12 = A12(:, 2:end) .* (1 ./ N);

    A21 = ifft(fft(Theta, [], 1), [], 2);
    A21 = A21(:, 2:end).' .* (1 ./ M);

    extVec = [0; firstCol];
    ind = triu(indicesM) + 1 - eye(M);
    A11 = reshape(extVec(ind), size(ind));
    extVec = [0; firstColI];
    ind = tril(indicesM) + 1;
    A11 = A11 + reshape(extVec(ind), size(ind));

    extVec = [0; firstRow(1:end - 1)];
    ind = triu(indicesN) + 1 - eye(N - 1);
    A22 = reshape(extVec(ind), size(ind));
    extVec = [0; firstRowI(1:end - 1)];
    ind = tril(indicesN) + 1;
    A22 = A22 + reshape(extVec(ind), size(ind));

    A = [A11, A12; A21, A22];
    yZero = [xiHandle(x), etaHandle(y(2:end))].';
    Phi0Theta = ifft2(Phi0 .* Theta);
    Phi0Theta = [Phi0Theta(:, 1); Phi0Theta(1, 2:end).'];
    B = yZero - Phi0Theta;
    phi0 = linsolve(A, B);
    phi(:, 1) = phi0(1:M);
    phi(1, 2:end) = phi0(M + 1: end).';

    Phi = fft2(phi);

    res = ifft2(Phi .* Theta);
end
        \end{verbatim}
    
\end{document}